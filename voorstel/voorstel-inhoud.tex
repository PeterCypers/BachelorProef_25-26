%---------- Inleiding ---------------------------------------------------------

% TODO: Is dit voorstel gebaseerd op een paper van Research Methods die je
% vorig jaar hebt ingediend? Heb je daarbij eventueel samengewerkt met een
% andere student?
% Zo ja, haal dan de tekst hieronder uit commentaar en pas aan.

%\paragraph{Opmerking}

% Dit voorstel is gebaseerd op het onderzoeksvoorstel dat werd geschreven in het
% kader van het vak Research Methods dat ik (vorig/dit) academiejaar heb
% uitgewerkt (met medesturent VOORNAAM NAAM als mede-auteur).
% 

\section{Inleiding}%
\label{sec:inleiding}

In moderne softwareontwikkeling zijn API's een cruciale schakel tussen systemen. Het ontwikkelen en testen van deze API's vereist echter veel repetitieve en technische handelingen zoals het schrijven van testcases, het valideren van responses, en het controleren van documentatie. In agile teams, waar snelheid en iteratie centraal staan, vormt dit een bottleneck. Er is nood aan een intelligente assistent die deze taken kan ondersteunen of deels automatiseren. De vraag rijst of een AI-agent, gebaseerd op recente ontwikkelingen in natural language processing en machine learning, hierin een rol kan spelen.

De onderzoeksvraag vanuit mijn stagebedrijf IT1 luidt als volgt: \emph{Hoe kan een AI-agent bijdragen aan het ondersteunen van het ontwikkel- en testproces van RESTful API's binnen een agile softwareontwikkeltraject?} 

Hieronder is het probleem opgesplitst in een aantal deelvragen die zullen bijdragen tot een uitwerking van de hoofdonderzoeksvraag:

\begin{itemize}
  \item Wat zijn de huidige uitdagingen bij het ontwikkelen en testen van RESTful API's?
  \item Welke taken binnen dit proces zijn het meest repetitief of foutgevoelig?
  \item Welke bestaande tools en frameworks worden momenteel gebruikt voor API
  testing en documentatie?
  \item Welke AI-technieken (bv. LLMs, NLP, code generation) zijn geschikt om deze taken te ondersteunen?
  \item Hoe kan een AI-agent geïntegreerd worden in een bestaande ontwikkelworkflow?
  \item Wat zijn de beperkingen en risico’s van het inzetten van een AI-agent in dit domein?
\end{itemize}

Het doel van dit onderzoek is om na te gaan in welke mate een AI-agent kan bijdragen aan het efficiënter ontwikkelen en testen van RESTful API's. Dit gebeurt door het analyseren van bestaande tools, het ontwikkelen van een proof-of-concept AI-agent en het evalueren van de impact ervan op het ontwikkelproces binnen een afgebakend scenario.

%---------- Stand van zaken ---------------------------------------------------

\section{Literatuurstudie}%
\label{sec:literatuurstudie}

Citation-test
\begin{itemize}
    \item \autocite{Ehsan2022}
    \item \autocite{Golmohammadi2023}
    \item \autocite{Kim2023}
    \item \autocite{NavarathnaMudiyanselage2024}
    \item \autocite{Fielding2000}
    \item \autocite{IBMTech2025}
    \item \autocite{Gutowska2025}
    \item \autocite{Martin-Lopez2022}
    \item \autocite{AnthropicMCP2025}
\end{itemize}

Hier beschrijf je de \emph{state-of-the-art} rondom je gekozen onderzoeksdomein, d.w.z.\ een inleidende, doorlopende tekst over het onderzoeksdomein van je bachelorproef. Je steunt daarbij heel sterk op de professionele \emph{vakliteratuur}, en niet zozeer op populariserende teksten voor een breed publiek. Wat is de huidige stand van zaken in dit domein, en wat zijn nog eventuele open vragen (die misschien de aanleiding waren tot je onderzoeksvraag!)?

Je mag de titel van deze sectie ook aanpassen (literatuurstudie, stand van zaken, enz.). Zijn er al gelijkaardige onderzoeken gevoerd? Wat concluderen ze? Wat is het verschil met jouw onderzoek?

Verwijs bij elke introductie van een term of bewering over het domein naar de vakliteratuur, bijvoorbeeld~\autocite{Hykes2013}! Denk zeker goed na welke werken je refereert en waarom.

Draag zorg voor correcte literatuurverwijzingen! Een bronvermelding hoort thuis \emph{binnen} de zin waar je je op die bron baseert, dus niet er buiten! Maak meteen een verwijzing als je gebruik maakt van een bron. Doe dit dus \emph{niet} aan het einde van een lange paragraaf. Baseer nooit teveel aansluitende tekst op eenzelfde bron.

Als je informatie over bronnen verzamelt in JabRef, zorg er dan voor dat alle nodige info aanwezig is om de bron terug te vinden (zoals uitvoerig besproken in de lessen Research Methods).

% Voor literatuurverwijzingen zijn er twee belangrijke commando's:
% \autocite{KEY} => (Auteur, jaartal) Gebruik dit als de naam van de auteur
%   geen onderdeel is van de zin.
% \textcite{KEY} => Auteur (jaartal)  Gebruik dit als de auteursnaam wel een
%   functie heeft in de zin (bv. ``Uit onderzoek door Doll & Hill (1954) bleek
%   ...'')

Je mag deze sectie nog verder onderverdelen in subsecties als dit de structuur van de tekst kan verduidelijken.

%---------- Methodologie ------------------------------------------------------
\section{Methodologie}%
\label{sec:methodologie}

In de initiele fase zal ik een literatuurstudie doen om inzichten te verwerven over het problematiek omlijnd in de deelvragen \& hoofdonderzoeksvraag in de inleiding. Ik zal de huidige state of the art, alsook de richting waarin we naartoe kunnen evolueren verkennen. De evolutie betreffende de mogelijkheden voor AI-support bij het ontwikkelen en testen van Restful APIs.

% TODO: in welke fase?

Betreffende AI-agents zal ik in de daaropvolgende fase onderzoeken welke AI model ik zal gebruiken, de mogelijke security risico's en kosten die erbij komen kijken. Hier zal ik ook heel wat onderzoek doen onder anderen over agentic design patronen, MCP \& RAG. Daarenboven zal ik ook onderzoeken in welke taal(en) en framework(s) ik de proof of concept zal uitwerken.

De POC AI-agent zal het volgende kunnen:
\begin{itemize}
    \item API-specificaties interpreteren (bvb.: Swagger/OpenAPI)
    \item Automatisch testcases genereren (bvb.: met Postman of via Python/JavaScript)
    \item Foutmeldingen analyseren en suggesties geven
    \item Documentatie aanvullen or genereren
\end{itemize}

In de volgende fase zullen deze functionaliteiten van de POC agent geïmplementeerd worden, er zal ook uitgezocht worden welke soort testen er kunnen worden gedaan en met welke tools de agent kan werken om die verschillende soorten testen te doen/ondersteunen.

%todo : welke metingen? welke technieken gebruik ik om die metingen te maken?
In de volgende fase zal ik bepaalde metingen doen met de POC agent die een mogelijke verbetering in efficiëntie van werken aan APIs moet kunnen aantonen

%todo: laatste fase
En een laatste fase verbonden met de resultaten...

%---------- Verwachte resultaten ----------------------------------------------
\section{Verwacht resultaat, conclusie}%
\label{sec:verwachte_resultaten}

Men verwacht dat de POC AI-Agent zal kunnen aantonen of er een meetbare verbetering is in efficiëntie van werken door het automatiseren van repetitieve taken bij het ontwikkelen en testen van RESTful APIs. Dat er kan aangetoond worden dat die automatisaties tijdsbesparend zullen zijn. Ook dat er weinig fouten zullen gebeuren bij het genereren van testcases. Er wordt een hoge test-coverage verwacht. Ook binnen de verwachtingen valt dat de agent zich houdt aan bepaalde best-practices en API-specificaties.

Dit onderzoek zal helpen om een recommandatie te kunnen maken m.b.t. het al dan niet gebruiken van een AI-Agent bij het werken aan APIs bij IT1 en ook wat inzicht bieden over de verschillende mogelijkheden binnen de wereld van agentic workflows.


