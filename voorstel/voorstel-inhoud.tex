%---------- Inleiding ---------------------------------------------------------

% TODO: Is dit voorstel gebaseerd op een paper van Research Methods die je
% vorig jaar hebt ingediend? Heb je daarbij eventueel samengewerkt met een
% andere student?
% Zo ja, haal dan de tekst hieronder uit commentaar en pas aan.

%\paragraph{Opmerking}

% Dit voorstel is gebaseerd op het onderzoeksvoorstel dat werd geschreven in het
% kader van het vak Research Methods dat ik (vorig/dit) academiejaar heb
% uitgewerkt (met medesturent VOORNAAM NAAM als mede-auteur).
% 

\section{Inleiding}%
\label{sec:inleiding}

In moderne softwareontwikkeling zijn API's een cruciale schakel tussen systemen. Het ontwikkelen en testen van deze API's vereist echter veel repetitieve en technische handelingen zoals het schrijven van testcases, het valideren van responses, en het controleren van documentatie. In agile teams, waar snelheid en iteratie centraal staan, vormt dit een bottleneck. Er is nood aan een intelligente assistent die deze taken kan ondersteunen of deels automatiseren. De vraag rijst of een AI-agent, gebaseerd op recente ontwikkelingen in natural language processing en machine learning, hierin een rol kan spelen.

De onderzoeksvraag vanuit mijn stagebedrijf IT1 luidt als volgt: \emph{Hoe kan een AI-agent bijdragen aan het ondersteunen van het ontwikkel- en testproces van RESTful API's binnen een agile softwareontwikkeltraject?} 

Hieronder is het probleem opgesplitst in een aantal deelvragen die zullen bijdragen tot een uitwerking van de hoofdonderzoeksvraag:

\begin{itemize}
  \item Wat zijn de huidige uitdagingen bij het ontwikkelen en testen van RESTful API's?
  \item Welke taken binnen dit proces zijn het meest repetitief of foutgevoelig?
  \item Welke bestaande tools en frameworks worden momenteel gebruikt voor API
  testing en documentatie?
  \item Welke AI-technieken (bv. LLMs, NLP, code generation) zijn geschikt om deze taken te ondersteunen?
  \item Hoe kan een AI-agent geïntegreerd worden in een bestaande ontwikkelworkflow?
  \item Wat zijn de beperkingen en risico’s van het inzetten van een AI-agent in dit domein?
\end{itemize}

Het doel van dit onderzoek is om na te gaan in welke mate een AI-agent kan bijdragen aan het efficiënter ontwikkelen en testen van RESTful API's. Dit gebeurt door het analyseren van bestaande tools, het ontwikkelen van een proof-of-concept AI-agent en het evalueren van de impact ervan op het ontwikkelproces binnen een afgebakend scenario.

%---------- Stand van zaken ---------------------------------------------------

\section{Literatuurstudie}%
\label{sec:literatuurstudie}

Citation-test
\begin{itemize}
    \item \autocite{Ehsan2022}
    \item \autocite{Golmohammadi2023}
    \item \autocite{Kim2023}
    \item \autocite{NavarathnaMudiyanselage2024}
    \item \autocite{Fielding2000}
    \item \autocite{IBMTech2025}
    \item \autocite{Gutowska2025}
    \item \autocite{Martin-Lopez2022}
    \item \autocite{AnthropicMCP2025}
\end{itemize}

\subsection{Wat zijn REST APIs?}
%todo: bijlezen van Fielding: focus op API disambiguation & REST principes
API staat voor \emph{Application Programming Interface}. Dit is een
software-interface die het mogelijk maakt dat twee
applicaties met elkaar kunnen communiceren. Dit in
tegenstelling tot een User Interface, die mensen met
software laat werken.

Volgens \textcite{Fielding2000} gebruikt het moderne internet HTTP als een network-based Application Programming Interface (API).

Rest is een architectuurstijl die staat voor \emph{Representative state transfer}. Dit omschrijft een reeks van constraints die een soort best practice opstellen voor het gebruiken van APIs. De term REST wordt toegewezen aan en komt uit het werk van Roy \textcite{Fielding2000}

%todo: fill in the 6 principles
\subsubsection{REST-Principes/-constraints:}

Wat volgt zijn de principes uitgelijnd in het werk van Roy \textcite{Fielding2000}:
\begin{enumerate}
    %todo: explain request-response under one of these items
    \item \textbf{Client-server} \\
    Client-server architectuur. Seperation of concerns is het principe achter deze constraint. Het scheiden van user interface (client) en data storage (server) zorgt voor verbeterde portability en scalability. Zo kunnen de twee onafhankelijk van elkaar evolueren.
    \item \textbf{Stateless} \\
    TODO
    \item \textbf{Cache} \\
    Cachable
    \item \textbf{Uniform Interface} \\
    TODO
    \item \textbf{Layered System} \\
    TODO
    \item \textbf{code-on-demand (optioneel)} \\
    TODO
\end{enumerate}

\subsubsection{RESTfulness}
Wanneer een API aan alle verplichte constraints voldoet wordt het RESTful genoemd. Als het ontwerp van een API de REST constraints volgt maar in overtreding valt van minstens 1 verplicht constraint kan het hooguit "REST-like" genoemd worden, het niet toepassen van HATEOAS is een mogelijke oorzaak hiervan 
%todo: add reference - see reminders


\subsection{Testen Van RESTful APIs}

\subsubsection{Soorten Testen}
\label{sec:soorten_testen}
Bij een recente survey van gebruikte test-technieken bij 92 wetenschappelijke artikels door \textcite{Golmohammadi2023} zijn er 8 vaak voorkomende soorten tests relevant voor APIs ontdekt:
System Testing / Security Testing /  Integration Testing /  Unit Testing /  Regression Testing / Robustness Testing /  Architecture Design Testing / 
Acceptance Testing


\subsubsection{Testing Tools}
Een niet exhaustieve lijst van tools die gebruikt worden om APIs te testen getroffen bij de survey van \textcite{Golmohammadi2023}:
RESTest / RestAssured / Swagger Schema Validator / Postman / Burp Suite / SoapUI / APIFuzzer

Er bestaan enorm veel tools die elk een specifieke oplossing bieden voor de uiteenlopende vereisten bij het testen van APIs. Bvb.: Postman, een client tool voor het manueel testen van APIs en het valideren van responses, Burp Suite om de security te testen. RestAssured gebruikt DSL(domain specific language) om integration testen voor Java applicaties te schrijven met leesbare syntax in de vorm van een logische given().when().then() ketting.

%todo: find & source PHP+Laravel testing tools

\subsubsection{Documentatie Tools}

Bij HoGent hebben we reeds Swagger\footnote{\href{https://swagger.io/}{https://swagger.io/}} gebruikt om interactief te interageren met API endpoints, deze tool kan gebruikt worden om de endpoints op te lijsten en de bijhorende documentatie weer te geven, de tool kan ook gebruikt worden om requests te sturen en de responses te bekijken. Ook status codes en eventuele error berichten kunnen hiermee gecontroleerd worden.

OpenAPI is de specificatie en Swagger is de tool om die specificatie te implementeren. Zo luidt de vergelijking van die twee termen volgens \autocite{Pinkham2017}

%todo: find & source PHP+Laravel documentation tools

\subsection{OpenAPI Specification(OAS)}

De OAS is meer dan een guideline, het is zowat dé conventie die gehanteerd wordt bij het ontwikkelen van API's. Deze omschrijft onder anderen wat de correcte structuur is van API-requests \& responses. Het is belangrijk dat één werkwijze wordt gevolgd door ontwikkelaars zodat de werking van API's voorspelbaar en gebruiksvriendelijk blijft. Deze volgt REST-principes maar wordt ook up-to-date gehouden door de OpenAPI Initiative onder de Linux Foundation. \autocite{OpenAPIInitiativea}
\begin{quotation}
    "The OpenAPI Specification (OAS) defines a standard, language-agnostic interface to HTTP APIs which allows both humans and computers to discover and understand the capabilities of the service without access to source code, documentation, or through network traffic inspection. When properly defined, a consumer can understand and interact with the remote service with a minimal amount of implementation logic.
    
    An OpenAPI Description can then be used by documentation generation tools to display the API, code generation tools to generate servers and clients in various programming languages, testing tools, and many other use cases."
\end{quotation}
\hfill \autocite{OpenAPIInitiative}

%todo: gebrik Golgohammadi insight #03

%todo: also refer to martin-lopez' findings as a possible time-consuming & difficult factor while developing APIs

\subsection{AI Agents}

%todo: brief disambigation of what AI means -> cold be from wiki or anywhere
%todo: brief disambiguation of LLMs
%todo: Link

\subsubsection{Integratie van LLM’s in projecten met API’s}

Hier beschrijf je de \emph{state-of-the-art} rondom je gekozen onderzoeksdomein, d.w.z.\ een inleidende, doorlopende tekst over het onderzoeksdomein van je bachelorproef. Je steunt daarbij heel sterk op de professionele \emph{vakliteratuur}, en niet zozeer op populariserende teksten voor een breed publiek. Wat is de huidige stand van zaken in dit domein, en wat zijn nog eventuele open vragen (die misschien de aanleiding waren tot je onderzoeksvraag!)?

Je mag de titel van deze sectie ook aanpassen (literatuurstudie, stand van zaken, enz.). Zijn er al gelijkaardige onderzoeken gevoerd? Wat concluderen ze? Wat is het verschil met jouw onderzoek?

Verwijs bij elke introductie van een term of bewering over het domein naar de vakliteratuur, bijvoorbeeld~\autocite{Hykes2013}! Denk zeker goed na welke werken je refereert en waarom.

Draag zorg voor correcte literatuurverwijzingen! Een bronvermelding hoort thuis \emph{binnen} de zin waar je je op die bron baseert, dus niet er buiten! Maak meteen een verwijzing als je gebruik maakt van een bron. Doe dit dus \emph{niet} aan het einde van een lange paragraaf. Baseer nooit teveel aansluitende tekst op eenzelfde bron.

Als je informatie over bronnen verzamelt in JabRef, zorg er dan voor dat alle nodige info aanwezig is om de bron terug te vinden (zoals uitvoerig besproken in de lessen Research Methods).

% Voor literatuurverwijzingen zijn er twee belangrijke commando's:
% \autocite{KEY} => (Auteur, jaartal) Gebruik dit als de naam van de auteur
%   geen onderdeel is van de zin.
% \textcite{KEY} => Auteur (jaartal)  Gebruik dit als de auteursnaam wel een
%   functie heeft in de zin (bv. ``Uit onderzoek door Doll & Hill (1954) bleek
%   ...'')

Je mag deze sectie nog verder onderverdelen in subsecties als dit de structuur van de tekst kan verduidelijken.

%---------- Methodologie ------------------------------------------------------
\section{Methodologie}%
\label{sec:methodologie}

In de initiële fase zal ik een literatuurstudie doen om inzichten te verwerven over het problematiek omlijnd in de deelvragen \& hoofdonderzoeksvraag in de inleiding. Ik zal de huidige state of the art, alsook de richting waarin we naartoe kunnen evolueren verkennen. De evolutie betreffende de mogelijkheden voor AI-support bij het ontwikkelen en testen van RESTful APIs.

% TODO: in welke fase?
In de volgende fase zal ik de stand van zaken onderzoeken wat betreft reeds beschikbare AI-agents op de markt, experimenteren met lokale LLMs die beschikbaar zijn met Ollama\footnote{\href{https://ollama.com/}{https://ollama.com/}}.

Voor het maken van mijn eigen AI-agent zal ik in de daaropvolgende fase onderzoeken welke AI model ik zal gebruiken, de mogelijke security risico's en kosten die erbij komen kijken. Hier zal ik ook heel wat onderzoek doen onder anderen over agentic design patronen, MCP \& RAG. Daarenboven zal ik ook onderzoeken in welke taal(en) en framework(s) ik de proof of concept zal uitwerken.

De POC AI-agent zal het volgende kunnen:
\begin{itemize}
    \item API-specificaties interpreteren (bvb.: Swagger/OpenAPI)
    \item Automatisch testcases genereren (bvb.: met Postman of via Python/JavaScript)
    \item Foutmeldingen analyseren en suggesties geven
    \item Documentatie aanvullen or genereren
\end{itemize}

In de volgende fase zullen deze functionaliteiten van de POC agent geïmplementeerd worden, er zal ook uitgezocht worden welke soort testen er kunnen worden gedaan en met welke tools de agent kan werken om die verschillende soorten testen te doen/ondersteunen. Van de relevante testmethoden \ref{sec:soorten_testen} zal er in de eerste plaats een focus zijn op unit testing, mogelijks uitbreiden naar integratie testing en
afhankelijk van feedback mogelijks verder uitbreiden naar andere testmethodes.

%todo : welke metingen? welke technieken gebruik ik om die metingen te maken?
In de volgende fase zal het correct werken van de verschillende functionaliteiten van de agent worden getoetst. Daarna zal
de agent vergeleken worden met de in-house test/documentatie technieken. Mogelijke metrics: test-coverage, foutdetectie en performantie. Dit zal een mogelijke verbetering in efficiëntie van de API development / testing workflow kunnen aantonen.


%---------- Verwachte resultaten ----------------------------------------------
\section{Verwacht resultaat, conclusie}%
\label{sec:verwachte_resultaten}

Men verwacht dat de POC AI-Agent zal kunnen aantonen of er een meetbare verbetering is in efficiëntie van werken door het automatiseren van repetitieve taken bij het ontwikkelen en testen van RESTful APIs. Dat er kan aangetoond worden dat die automatisaties tijdsbesparend zullen zijn. Ook dat er weinig fouten zullen gebeuren bij het genereren van testcases. Er wordt een hoge test-coverage verwacht. Ook binnen de verwachtingen valt dat de agent zich houdt aan bepaalde best-practices en API-specificaties.

Dit onderzoek zal helpen om een recommandatie te kunnen maken m.b.t. het al dan niet gebruiken van een AI-Agent bij het werken aan APIs bij IT1 en ook wat inzicht bieden over de verschillende mogelijkheden binnen de wereld van agentic workflows.

\onecolumn
\section{Mindmap}%
\label{sec:Mindmap}
\begin{tikzpicture}
    \path [
    mindmap,
    text = white,
    level 1 concept/.append style =
    {font=\small\bfseries, sibling angle=90},
    level 2 concept/.append style =
    {font=\small\bfseries},
    level 3 concept/.append style =
    {font=\small\bfseries},
    BP/.style     = {concept, ball color=blue,
        font=\Huge\bfseries},
    Agents/.style = {concept, ball color=green!50!black},
    APIs/.style = {concept, ball color=blue!50!black},
    LLMs/.style = {concept, ball color=red!90!black},
    OpenAPI/.style = {concept, ball color=orange!90!black}
    ]
    node [BP] {BP} [clockwise from=0]
    child[concept color=green!50!black, nodes={Agents}] {
        node {Agents} [clockwise from=90]
        child { node {Types} }
        child { node {MCP} }
        child { node {RAG} }
        child { node {Workflow} }}
    child [concept color=blue, nodes={APIs}] {
        node {APIs} [clockwise from=300]
        child { node {Docu.} }
        child { node {Testen} }
        child { node {REST} }}
    child [concept color=red, nodes={LLMs}] {
        node {LLMs} [clockwise from=210]
        child { node {Modellen} [clockwise from=300]
            child { node {X vs Y} }}
        child { node {Prompts} [clockwise from=60]
            child { node {Tokens} }}}
    child [concept color=orange, nodes={OpenAPI}] {
        node {OpenAPI} };
\end{tikzpicture}

Mindmap van Stefan Kottwitz: \href{https://www.packtpub.com/hardware-and-creative/latex-cookbook}{https://www.packtpub.com/hardware-and-creative/latex-cookbook}
%\newpage
%\pagebreak



