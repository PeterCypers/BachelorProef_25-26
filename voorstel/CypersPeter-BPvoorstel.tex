%==============================================================================
% Sjabloon onderzoeksvoorstel bachproef
%==============================================================================
% Gebaseerd op document class `hogent-article'
% zie <https://github.com/HoGentTIN/latex-hogent-article>

% Voor een voorstel in het Engels: voeg de documentclass-optie [english] toe.
% Let op: kan enkel na toestemming van de bachelorproefcoördinator!
\documentclass{hogent-article}

% Preamble imports
\usepackage{hyperref}
\usepackage[landscape]{geometry}
\usepackage{tikz}
\usetikzlibrary{mindmap}
\usepackage{metalogo}

% Invoegen bibliografiebestand
\addbibresource{voorstel.bib}

% Informatie over de opleiding, het vak en soort opdracht
\studyprogramme{Professionele bachelor toegepaste informatica}
\course{Bachelorproef}
\assignmenttype{Onderzoeksvoorstel}
% Voor een voorstel in het Engels, haal de volgende 3 regels uit commentaar
% \studyprogramme{Bachelor of applied information technology}
% \course{Bachelor thesis}
% \assignmenttype{Research proposal}

\academicyear{2025-2026} % TODO: pas het academiejaar aan

\title{Onderzoek naar de inzet van een AI-agent ter ondersteuning van het ontwikkelen en testen van RESTful API's in een agile ontwikkelomgeving bij IT1}

\author{Peter Cypers}
\email{peter.cypers@student.hogent.be}

% Gaat het om een bachelorproef in samenwerking met een student in een andere
% opleiding? Geef dan de naam en emailadres hier
% \author{Yasmine Alaoui (naam opleiding)}
% \email{yasmine.alaoui@student.hogent.be}

%TODO: voeg promotor toe?
\supervisor[Co-promotor]{Toon De Groote (IT1, \href{mailto:t.degroote@webrand.be}{t.degroote@webrand.be})}

% Binnen welke specialisatierichting uit 3TI situeert dit onderzoek zich?
% Kies uit deze lijst:
%
% - Mobile \& Enterprise development
% - AI \& Data Engineering
% - Functional \& Business Analysis
% - System \& Network Administrator
% - Mainframe Expert
% - Als het onderzoek niet past binnen een van deze domeinen specifieer je deze
%   zelf
%
\specialisation{Mobile \& Enterprise development}
\keywords{AI Agent, Agentic Programming, REST API, API Testing, Test Automation, Large Language Models (LLMs), Natural Language Processing (NLP), Model Context Protocol (MCP), Retrieval-Augmented Generation (RAG)}

\begin{document}
%TODO Methodologie verder uitwerken
\begin{abstract}
  In deze bachelorproef onderzoeken we de mogelijkheden tot automatisatie van de ontwikkel- en testprocessen bij API’s, en in hoeverre dit kan ondersteund worden door een AI-agent. Het bedrijf IT1 nodigt me uit om een proof-of-concept(POC) A.I. agent te ontwikkelen. Ik zal een keuze maken uit beschikbare agent development mogelijkheden, er bestaan no-/lowcode agent build platvormen en SDK's(software development kits) die ik kan gebruiken. Mogelijke opties verkennen omtrent het gebruik van MCP(Model Context Protocol)-tools ...(verbinden met AI-apis? bvb Anthropic) [todo] [... Methodologie (1: hoe-AGENT maken/gebruiken) \& (2:gebruikte tech.) \& (3:hoe success meten) ... ] De agent zal belangrijke taken kunnen uitvoeren zoals het interpreteren van API-specificaties, het automatisch genereren van testcases, analyseren van foutmeldingen en eventueel documentatie aanvult of genereert. De uitwerking van de POC AI-agent zal aanwijzen of het inzetten van een gelijkaardige AI-agent de efficientie van het API development process al dan niet kan optimaliseren.
\end{abstract}

\tableofcontents

% De hoofdtekst van het voorstel zit in een apart bestand, zodat het makkelijk
% kan opgenomen worden in de bijlagen van de bachelorproef zelf.
%---------- Inleiding ---------------------------------------------------------

% TODO: Is dit voorstel gebaseerd op een paper van Research Methods die je
% vorig jaar hebt ingediend? Heb je daarbij eventueel samengewerkt met een
% andere student?
% Zo ja, haal dan de tekst hieronder uit commentaar en pas aan.

%\paragraph{Opmerking}

% Dit voorstel is gebaseerd op het onderzoeksvoorstel dat werd geschreven in het
% kader van het vak Research Methods dat ik (vorig/dit) academiejaar heb
% uitgewerkt (met medesturent VOORNAAM NAAM als mede-auteur).
% 

\section{Inleiding}%
\label{sec:inleiding}

In moderne softwareontwikkeling zijn APIs een cruciale schakel tussen systemen. Het ontwikkelen en testen van deze APIs vereist echter veel repetitieve en technische handelingen, zoals het schrijven van testcases, het valideren van responses, en het controleren van documentatie. In agile teams, waar snelheid en iteratie centraal staan, vormt dit een bottleneck. Er is nood aan een intelligente assistent die deze taken kan ondersteunen of deels automatiseren. De vraag rijst of een AI-agent, gebaseerd op recente ontwikkelingen in natural language processing en machine learning, hierin een rol kan spelen.

De onderzoeksvraag vanuit mijn stagebedrijf IT1 luid als volgt: \emph{Hoe kan een AI-agent bijdragen aan het ondersteunen van het ontwikkel- en testproces van RESTful APIs binnen een agile softwareontwikkeltraject?} 

Hieronder is het probleem opgesplitst in een aantal deelvragen die zullen bijdragen tot een uitwerking van de hoofdonderzoeksvraag:

\begin{itemize}
  \item Wat zijn de huidige uitdagingen bij het ontwikkelen en testen van RESTful APIs?
  \item Welke taken binnen dit proces zijn het meest repetitief of foutgevoelig?
  \item Welke bestaande tools en frameworks worden momenteel gebruikt voor API
  testing en documentatie?
  \item Welke AI-technieken (bv. LLMs, NLP, code generation) zijn geschikt om deze taken te ondersteunen?
  \item Hoe kan een AI-agent geïntegreerd worden in een bestaande ontwikkelworkflow?
  \item Wat zijn de beperkingen en risico’s van het inzetten van een AI-agent in dit domein?
\end{itemize}

Het doel van dit onderzoek is om na te gaan in welke mate een AI-agent kan bijdragen aan het efficiënter ontwikkelen en testen van RESTful APIs. Dit gebeurt door het analyseren van bestaande tools, het ontwikkelen van een proof-of-concept AI-agent, en het evalueren van de impact ervan op het ontwikkelproces binnen een afgebakend scenario.

%---------- Stand van zaken ---------------------------------------------------

\section{Literatuurstudie}%
\label{sec:literatuurstudie}

Citation-test
\begin{itemize}
    \item \autocite{Ehsan2022}
    \item \autocite{Golmohammadi2023}
    \item \autocite{Kim2023}
    \item \autocite{NavarathnaMudiyanselage2024}
    \item \autocite{Fielding2000}
    \item \autocite{IBMTech2025}
    \item \autocite{Gutowska2025}
    \item \autocite{Martin-Lopez2022}
    \item \autocite{AnthropicMCP2025}
\end{itemize}

Hier beschrijf je de \emph{state-of-the-art} rondom je gekozen onderzoeksdomein, d.w.z.\ een inleidende, doorlopende tekst over het onderzoeksdomein van je bachelorproef. Je steunt daarbij heel sterk op de professionele \emph{vakliteratuur}, en niet zozeer op populariserende teksten voor een breed publiek. Wat is de huidige stand van zaken in dit domein, en wat zijn nog eventuele open vragen (die misschien de aanleiding waren tot je onderzoeksvraag!)?

Je mag de titel van deze sectie ook aanpassen (literatuurstudie, stand van zaken, enz.). Zijn er al gelijkaardige onderzoeken gevoerd? Wat concluderen ze? Wat is het verschil met jouw onderzoek?

Verwijs bij elke introductie van een term of bewering over het domein naar de vakliteratuur, bijvoorbeeld~\autocite{Hykes2013}! Denk zeker goed na welke werken je refereert en waarom.

Draag zorg voor correcte literatuurverwijzingen! Een bronvermelding hoort thuis \emph{binnen} de zin waar je je op die bron baseert, dus niet er buiten! Maak meteen een verwijzing als je gebruik maakt van een bron. Doe dit dus \emph{niet} aan het einde van een lange paragraaf. Baseer nooit teveel aansluitende tekst op eenzelfde bron.

Als je informatie over bronnen verzamelt in JabRef, zorg er dan voor dat alle nodige info aanwezig is om de bron terug te vinden (zoals uitvoerig besproken in de lessen Research Methods).

% Voor literatuurverwijzingen zijn er twee belangrijke commando's:
% \autocite{KEY} => (Auteur, jaartal) Gebruik dit als de naam van de auteur
%   geen onderdeel is van de zin.
% \textcite{KEY} => Auteur (jaartal)  Gebruik dit als de auteursnaam wel een
%   functie heeft in de zin (bv. ``Uit onderzoek door Doll & Hill (1954) bleek
%   ...'')

Je mag deze sectie nog verder onderverdelen in subsecties als dit de structuur van de tekst kan verduidelijken.

%---------- Methodologie ------------------------------------------------------
\section{Methodologie}%
\label{sec:methodologie}

Hier beschrijf je hoe je van plan bent het onderzoek te voeren. Welke onderzoekstechniek ga je toepassen om elk van je onderzoeksvragen te beantwoorden? Gebruik je hiervoor literatuurstudie, interviews met belanghebbenden (bv.~voor requirements-analyse), experimenten, simulaties, vergelijkende studie, risico-analyse, PoC, \ldots?

Valt je onderwerp onder één van de typische soorten bachelorproeven die besproken zijn in de lessen Research Methods (bv.\ vergelijkende studie of risico-analyse)? Zorg er dan ook voor dat we duidelijk de verschillende stappen terug vinden die we verwachten in dit soort onderzoek!

Vermijd onderzoekstechnieken die geen objectieve, meetbare resultaten kunnen opleveren. Enquêtes, bijvoorbeeld, zijn voor een bachelorproef informatica meestal \textbf{niet geschikt}. De antwoorden zijn eerder meningen dan feiten en in de praktijk blijkt het ook bijzonder moeilijk om voldoende respondenten te vinden. Studenten die een enquête willen voeren, hebben meestal ook geen goede definitie van de populatie, waardoor ook niet kan aangetoond worden dat eventuele resultaten representatief zijn.

Uit dit onderdeel moet duidelijk naar voor komen dat je bachelorproef ook technisch voldoen\-de diepgang zal bevatten. Het zou niet kloppen als een bachelorproef informatica ook door bv.\ een student marketing zou kunnen uitgevoerd worden.

Je beschrijft ook al welke tools (hardware, software, diensten, \ldots) je denkt hiervoor te gebruiken of te ontwikkelen.

Probeer ook een tijdschatting te maken. Hoe lang zal je met elke fase van je onderzoek bezig zijn en wat zijn de concrete \emph{deliverables} in elke fase?

%---------- Verwachte resultaten ----------------------------------------------
\section{Verwacht resultaat, conclusie}%
\label{sec:verwachte_resultaten}

Hier beschrijf je welke resultaten je verwacht. Als je metingen en simulaties uitvoert, kan je hier al mock-ups maken van de grafieken samen met de verwachte conclusies. Benoem zeker al je assen en de onderdelen van de grafiek die je gaat gebruiken. Dit zorgt ervoor dat je concreet weet welk soort data je moet verzamelen en hoe je die moet meten.

Wat heeft de doelgroep van je onderzoek aan het resultaat? Op welke manier zorgt jouw bachelorproef voor een meerwaarde?

Hier beschrijf je wat je verwacht uit je onderzoek, met de motivatie waarom. Het is \textbf{niet} erg indien uit je onderzoek andere resultaten en conclusies vloeien dan dat je hier beschrijft: het is dan juist interessant om te onderzoeken waarom jouw hypothesen niet overeenkomen met de resultaten.



\printbibliography[heading=bibintoc]

\end{document}